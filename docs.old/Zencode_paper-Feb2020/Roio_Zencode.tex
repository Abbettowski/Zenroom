%%%%%%%%%%%%%%%%%%%% author.tex %%%%%%%%%%%%%%%%%%%%%%%%%%%%%%%%%%%
%
% sample root file for your "contribution" to a proceedings volume
%
% Use this file as a template for your own input.
%
%%%%%%%%%%%%%%%% Springer %%%%%%%%%%%%%%%%%%%%%%%%%%%%%%%%%%


\documentclass{svproc}
%
% RECOMMENDED %%%%%%%%%%%%%%%%%%%%%%%%%%%%%%%%%%%%%%%%%%%%%%%%%%%
%

% to typeset URLs, URIs, and DOIs
\usepackage{url}
\usepackage{csquotes}
\usepackage{cite}
\def\UrlFont{\rmfamily}

\begin{document}
\mainmatter              % start of a contribution
%
\title{Zencode DSL: human language smart contracts}
%
\titlerunning{Zencode}
% abbreviated title (for running head)
%                                     also used for the TOC unless
%                                     \toctitle is used
%
\author{Denis Roio\inst{1}}
%
\authorrunning{Denis Roio} % abbreviated author list (for running head)
%
%%%% list of authors for the TOC (use if author list has to be modified)
\tocauthor{Denis Roio}
%
\institute{Dyne.org foundation, Amsterdam,\\
\email{jaromil@dyne.org},\\ WWW home page:
\texttt{https://dyne.org}}

\maketitle              % typeset the title of the contribution

\begin{abstract}
% 70 to 150 words

  This paper presents a theoretical framework and a software solution
  to facilitate technological sovereignty and data commons management.

  The goal is to improve people's awareness of how their data is
  processed by algorithms, as well facilitate developers to write
  applications that follow privacy by design principles.

  The main requirement is that of distributed computing: Zencode must
  be capable of processing un/trusted code to execute advanced
  cryptographic functions, to be with any blockchain implementation as
  an interpreter of smart contracts.

  Zencode is language to write portable scripts executed inside an
  isolated environment (the Zenroom VM) that can be ported to any
  platform, embedded in any language and made inter-operable with any
  blockchain.

  The Zencode implementation is inspired by research on
  language-theoretical security, adopts Lua as direct-syntax parser
  to build a non-Turing complete domain-specific language (DSL)
  enforcing coarse-grained computations and recognition of data before
  processing.

% We would like to encourage you to list your keywords within
% the abstract section using the \keywords{...} command.
\keywords{blockchain, language, smart-contracts, dsl, langsec}
\end{abstract}
%
\section{Introduction}
%
Since DECODE project's inception, developing the Zencode language and
releasing the Zenroom VM interpreter has been an extremely motivating
ambition, as it concretely provides a solution for the
techno-political implications illustrated by the AlgoSov.org
observatory and researched in my Ph.D thesis on "Algorithmic
Sovereignty".

I begin this paper illustrating the techno-political motivations for
the development of Zenroom in the context of the DECODE project, an
European H2020 grant (nr. 732546) coordinated by colleague
Dr. Francesca Bria as its principal investigator.

I'll then proceed sharing my considerations on the state of the art of
language design and security of execution in trust-less
environments. The safe execution of untrusted code is required by most
distributed ledger technologies (also commonly referred to as
blockchain); it is as well a desirable feature for the reliability of
cryptographic data manipulation for general use (certification,
authentication and more advanced uses contemplated in Zenroom).

At last this paper consists a brief introduction of the Zencode DSL
design and points to the Zenroom VM interpreter implementation to
execute safely and efficiently simplified smart-rules describing
cryptographic operation and data transformations using human readable
language.

%
\subsection{For the awareness of algorithms}
%

The goal of the Zenroom VM and the Zencode language is ultimately that
of realizing a simple, non-technical, human-readable language for
smart-rules that are actually executed in a verifiable and provable
manner within a controlled execution environment.

To articulate the importance of this quest and the relevance of the
results presented, which I believe to be unique in the landscape of
blockchain smart-contract languages, is important to remind us of the
condition in which most people find themselves when participating in
the regime of truth that is built by algorithms.

As the demand and production of well-connected vessels for the digital
dimension has boomed, machine-readable code today functions as a
literature informing the architecture in which human interactions
happens and decisions are taken. The "telematic condition" is realized
by an integrated data-work continuously engaging the observer as a
participant. Such a "Gesamtdatenwerk" \cite{ascott2004} may seem
an abstract architecture, yet it can be deeply binding under legal,
ethical and moral circumstances.

The comprehension of algorithms, the awareness of the way decisions
are formulated, the implications of their execution, is not just a
technical condition, but a political one, for which access to
information cannot be just considered a feature, but a civil right
\cite{pelizza2017a}. It is important to understand this in
relation to the "classical" application of algorithms executed in a
centralized manner, but even more in relation to distributed computing
scenarios posed by blockchain technologies, which theorize a future in
which rules and contracts are executed irrevocably and without
requiring any human agency.

The legal implications with regards to standing rights and liabilities
are out of the scope here, while the focus is on ways humans, even
when lacking technical literacy, can be made aware of what an
algorithm does. Is it possible to establish the ground for a shared
language that informs digital architects about their choices and
inhabitants about the digital territory? Going past assumptions about
the strong role algorithms have in governance and accountability
\cite{diakopoulos2016a}, how can we inform digital citizens about
their condition?  When describing the virtualization of economic
activity in the global context, Saskia Sassen describes the need we
are observing as that of an analytical vocabulary:

\blockquote{The third component in the new geography of power is the
  growing importance of electronic space. There is much to be said on
  this issue. Here, I can isolate one particular matter: the
  distinctive challenge that the virtualization of a growing number of
  economic activities presents not only to the existing state
  regulatory apparatus, but also to private-sector institutions
  increasingly dependent on the new technologies. Taken to its
  extreme, this may signal a control crisis in the making, one for
  which we lack an analytical vocabulary. \cite{sassen1996a} }

The analysis of legal texts and regulations here shifts into an
entirely new domain; it has to refer to conditions that only
algorithms can help build or destroy. Thus, referring to this
theoretical framework, the research and development of a free and open
source language that is intelligible to humans becomes of crucial
importance and, from an ethical standing point, DECODE as many other
projects in the same space cannot be exempted from addressing it.

When we consider algorithms as contracts regulating relationships
(between humans, between humans and nature and, nowadays more
increasingly, between different contexts of nature itself) then we
should adopt a representation that is close to how the human mind
works and that is directly connected to the language adopted. Since
algorithms are the systemic product of complex relationships between
contracts and relevant choices made by standing actors
\cite{monico2014}, the ability to verify which algorithms are in place
for a certain result to be visualized becomes very important and
should be embedded in every application: to understand and communicate
what algorithms and to describe and experiment their repercussions on
reality.

For a deeper exploration of the techno-political implications raised
by this document please refer to DECODE's blog-post on Algorithmic
Sovereignty which also contains a series of historical examples of
critical situations that help to understand the urgency we are facing.

DECODE goes in the direction of following a technical and scienifical
restearch path and call for a new form of municipal rationality that
contemplates technological sovereignty, citizen participation and
ownership.

This narrative is echoing through world's biggest municipal
administrations as we speak: a stance against the colonization of
dense settlements by complex technical systems that are far from the
reach of citizen's political control. The "Manifesto in favour of
technological sovereignty and digital rights for cities" is now being
considered as a standard guideline for ethics in governance by many
cities of the world.

This whitepaper is then also a call for action to fellow programmers
out there: we need to write code that is understandable by other
humans and by animals, plants, all the living world we inoculate with
our sensors and manipulate through automation. The term "smart" should
really mean understandable, accessible, open and trustworthy
\cite{nevejan2007}; then smart-contracts should be expressed in a
language that most humans can understand. Good code is not what is
skillfully crafted or most efficient, but what can be read by others,
studied, changed, adapted.

Let's adopt intuitive name-spaces that can be easily matched with
reality or simple metaphors, let's make sure that what we write is
close to what we mean. Common understanding of algorithms is
necessary, because their governance is an inter-disciplinary exercise
and cannot be left in the hands of a technical elite.

%
\section{Language Security}
%

This section will establish the underpinnings of the Zencode language,
starting from its most theoretical assumptions, to conclude with
specific requirements. In order to do so, I will concentrate on the
recent corpus developed by research on "language-theoretic security"
(LangSec). Here below we include a brief explanation condensed from
the information material of the LangSec.org project hosted at
IEEE. This research benefits from being informed by the experience of
the exploit development community: exploitation is a practical
exploration of the space of unanticipated state, its prevention or
containment.

\blockquote{In a nutshell [...] LangSec is the idea that many security
  issues can be avoided by applying a standard process to input
  processing and protocol design: the acceptable input to a program
  should be well-defined (i.e., via a grammar), as simple as possible
  (on the Chomsky scale of syntactic complexity), and fully validated
  before use (by a dedicated parser of appropriate but not excessive
  power in the Chomsky hierarchy of
  automata). \cite{DBLP:conf/secdev/MomotBHP16} }

LangSec is a design and programming philosophy that focuses on
formally correct and verifiable input handling throughout all phases
of the software development lifecycle. In doing so, it offers a
practical method of assurance of software free from broad and
currently dominant classes of bugs and vulnerabilities related to
incorrect parsing and interpretation of messages between software
components (packets, protocol messages, file formats, function
parameters, etc.).

This design and programming paradigm begins with a description of
valid inputs to a program as a formal language (such as a
grammar). The purpose of such a disciplined specification is to
cleanly separate the input-handling code and processing code. A
LangSec-compliant design properly transforms input-handling code into
a recognizer for the input language; this recognizer rejects
non-conforming inputs and transforms conforming inputs to structured
data (such as an object or a tree structure, ready for type- or
value-based pattern matching). The processing code can then access the
structured data (but not the raw inputs or parsers temporary data
artifacts) under a set of assumptions regarding the accepted inputs
that are enforced by the recognizer.

This approach leads to several advantages:
\begin{enumerate}
\item produce verifiable recognizers, free of typical classes of ad-hoc parsing bugs
\item produce verifiable, composable implementations of distributed systems that ensure equivalent parsing of messages by all components and eliminate exploitable differences in message interpretation by the elements of a distributed system
\item mitigate the common risks of ungoverned development by
  explicitly exposing the processing dependencies on the parsed input.
\end{enumerate}

As a design philosophy, LangSec focuses on a particular choice of
verification trade-offs: namely, correctness and computational
equivalence of input processors.

\subsection{Ad-hoc notions of input validity}

Formal verification of input handlers is impossible without formal
language-theoretic specification of their inputs, whether these inputs
are packets, messages, protocol units, or file formats. Therefore,
design of an input-handling program must start with such a formal
specification.  Once specified, the input language should be reduced
to the least complex class requiring the least computational power to
recognize. Considering the tendency of hand-coded programs to admit
extra state and computation paths, computational power susceptible to
crafted inputs should be minimized whenever possible. Whenever the
input language is allowed to achieve Turing-complete power, input
validation becomes undecidable; such situations should be avoided.

\subsection{Parser differentials}

Mutual misinterpretation between system components. Verifiable
composition is impossible without the means of establishing parsing
equivalence between different components of a distributed
system. Different interpretation of messages or data streams by
components breaks any assumptions that components adhere to a shared
specification and so introduces inconsistent state and unanticipated
computation \cite{DBLP:conf/secdev/MomotBHP16}. In addition, it breaks
any security schemes in which equivalent parsing of messages is a
formal requirement, such as the contents of a certificate or of a
signed message being interpreted identically, for example a X.509
Certificate Signing Request as seen by a Certificate Authority vs. the
signed certificates as seen by the clients or signed app package
contents as seen by the signature verifier versus the same content as
seen by the installer (as in the recent Android Master Key bug
\cite{freeman2013exploit}. An input language specification stronger
than deterministic context-free makes the problem of establishing
parser equivalence undecidable. Such input languages and systems whose
trustworthiness is predicated on the component parser equivalence
should be avoided. Logical programming using Prolog for instance, or
languages like Scheme derived from LISP, or OCaml or Erlang would
match then our requirements, but they aren't as usable as desired. As
a partial solution to this problem the Zencode language parser (and
all its components and eventually linked shared libraries) should be
small, portable, self-contained and clearly versioned with a
verifiable hash.

\subsection{Mixing of input recognition and processing}

Mixing of basic input validation ("sanity checks") and logically
subsequent processing steps that belong only after the integrity of
the entire message has been established makes validation hard or
impossible. As a practical consequence, unanticipated reachable state
exposed by such premature optimization explodes. This explosion makes
principled analysis of the possible computation paths
untenable. LangSec-style separation of the recognizer and processor
code creates a natural partitioning that allows for simpler
specification-based verification and management of code. In such
designs, effective elimination of exploit-enabling implicit data flows
can be achieved by simple systems memory isolation primitives.

\subsection{Language specification drift}

A common practice encouraged by rapid software development is the
unconstrained addition of new features to software components and
their corresponding reflection in input language
specifications. Expressing complex ideas in hastily written code is a
hallmark of such development practices. In essence, adding new input
feature requirements to an already underspecified input language
compounds the explosion of state and computational paths.

%
\section{The Zencode Language}
%

This section describes the salient implementation details of the
Zencode DSL, the smart-rule language for DECODE, tailored on its
use-cases and based on the Zenroom controlled execution environment
(VM).

This section consists of three parts, each one explaining:
\begin{itemize}
\item the language model inherited from BDD / Cucumber
\item the data validation model based on schema validation
\item the memory model for safe computation
\end{itemize}

\subsection{Syntax-Directed Translation}

Lua is an interpreted, cross-platform, embeddable, performant and
low-footprint language. Lua's popularity is on the rise in the last
couple of years \cite{costin2017lua}. Simple design and efficient
usage of resources combined with its performance make it attractive
for production web applications, even to big organizations such as
Wikipedia, CloudFlare and GitHub. In addition to this, Lua is one of
the preferred choices for programming embedded and IoT devices. This
context allows an assumption of a large and growing Lua codebase yet
to be assessed. This growing Lua codebase could be potentially driving
production servers and an extremely large number of devices, some
perhaps with mission-critical function for example in automotive or
home-automation domains.

Lua stability has been extensively tested through a number of public
applications including the adoption by the gaming industry for
untrusted language processing in "World of Warcraft" scripting. It is
ideal for implementing an external DSL using C99 as a host language.

\subsection{Behaviour Driven Development}

In Behaviour Driven Development (BDD), the important role of software
integration and unit tests is extended to serve both the purposes of
designing the human-machine interaction flow (user journey in UX
terms) and of laying down a common ground for interaction between
designers and stakeholders. In this Agile software development
methodology the software testing suite is based on natural language
units that grant a common understanding for all participants and
observers.

I'm very grateful to my friend and colleague Puria Nafisi Azizi for
this brilliant intuition: adopting BDD for developing Zencode and
implement a human-friendly language that does not depends on the
underlying cryptographic implementation, allowing to share simple
knowledge on how to include crypto scenarios components in different
applications as well how to update them..

For our implementation of Zencode, definable as a dialect of BDD, the
first step has been that of mapping series of interconnected cascading
sentences of operations to the actual source code describing their
execution to the Zenroom VM; this implementation has to be done
manually with knowledge of Lua scripting and of the higher level
functions that grant communication with the Zenroom VM.

Zencode then becomes a "textual frontend" that is easy to embed in
graphical applications and whose purpose is to wire expressions and
executions by means of utterances expressed in human language.

Referring to the Cucumber implementation of BDD, arguably the most
popular in use by the industry to day and factual standard
\cite{wynne2012}, the grammar of utterances is very simple and
definable as a "cascading" flow indeed, since the fixed sequence of
lines can follow only one fixed order:

\enquote{Given .. and* .. When .. and* .. Then print ..}

This sequence is fixed and in simple terms consists of:

\begin{enumerate}
\item an read-only initialisation of states "Given (and)"
\item an read-write scenario based transformation of states "When (and)"
\item a write-only publishing phase of final states "Then (and)".
\end{enumerate}

The Zenroom implementation simply defines fixed sequences of strings, mapping them to cryptographic functions, allowing the presence of variables that are expected to be arguments for the functions. These variables can then be changed by participants (frontend developers or application operators) as they are marked by inclusion a repeating sequence of two adjacent single quotes ('  ').

The underlying parser is based on a finite state machine controlling
the change of states and capable of executing security operations:
data validation checks and memory wiping.

Zencode acts upon a positive, unique and non-flexible match of the
first word of each new line, checks it complies with the current
parser machine state and then proceeds parsing the whole phrase minus
the variables, saving a pointer to the corresponding function if found
along with the contents of variables if any.

As a result, one (or more, synonyms are supported) non-repeating line
of parsed Zencode utterance is easy to translate across different
spoken languages and corresponds to a declared function allowing the
execution of Lua commands inside the Zenroom VM.

The current implementation addresses specific scenarios useful to the
pilots in DECODE, while contemplating future extensions. Scenarios
available:

\begin{itemize}
\item Simple symmetric transformations of cipher-text by means of HASH and KDF transformations
\item Diffie-Hellman asymmetric key encryption (AES-GCM)
\item Zero-Knowledge proof and blind-signing of credentials for unlinkable selective attribute revelations
\end{itemize}

Documentation, examples and an interactive online webassembly demo
are all available online on the website dev.zenroom.org.

\subsection{Declarative Schema Validation}

In order to make the processing of Zencode more robust, all data used
as input and output for its computations is validated according to
predefined schemas. This makes the Zencode DSL a declarative language
in which data recognition is operated before processing.

The data schemas are added on a per-usecase basis: they refer to
specific cryptographic implementations as they are added in
Zencode. Careful evaluation regarding their addition is made to
realize if old schemas can be extended to include new requirements.

Schemas are expressed in a simple format using Lua scripting syntax
and consist of:
\begin{itemize}
\item an importer from JSON data structures containing hex or base64 encoded complex data types
\item an exporter of complex structured data types to big numbers encoded using hex or base64 and other common encoding formats
\end{itemize}

Every data structure processed in Zencode enters as a JSON or CBOR
string input (IN), it is decoded and parsed, then checked for
cryptographic validity (for instance checking point-on-curve) and
stored in its validated data type (ACK) and at last is encoded back
from defined data types to JSON or CBOR output string using encoding
methods (OUT).

This creates three cascading sections in the HEAP of Zenroom and each
section corresponds to the language steps in Zencode:

\begin{enumerate}
\item Given (IN)
\item When (ACK)
\item Then (OUT)
\end{enumerate}

Providing a rigid structure to context-specific (or pilot-specific)
implementations of Zencode scenarios: the parser should always operate
data recognition in the Given/IN phase, operate transformations in the
When/ACK phase and finally render output in the Then/OUT phase. This
flow is locked with recurring HEAP checks to insure that different
areas of memory are not accessed by the wrong section of Zencode, as
well thatn the When/ACK phase is operated only on decoded memory and
verified schemas.

\section{Conclusion}

This brief paper serves as an introduction to the motivation and
design choices behind the Zencode DSL and, only partially, to the
Zenroom VM. The production-ready implementation of Zencode should be
seen as complementary to this paper and is publicly available under
Affero General Public License v3 from the website Zenroom.org.

% ---- Bibliography ----
%
\bibliography{Roio_Zencode}{}
\bibliographystyle{plain}
\end{document}
